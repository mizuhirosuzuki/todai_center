\documentclass[12pt,letterpaper]{article}
\usepackage[utf8]{inputenc}
\usepackage{float}
\usepackage{pdflscape}
\usepackage{amsmath}
\usepackage{amsfonts}
\usepackage{amssymb}
\usepackage{fullpage}
\usepackage{gensymb, comment}
\usepackage{textcomp}
\usepackage{threeparttable}
\usepackage{graphicx}
\usepackage[sort]{natbib}
\usepackage{setspace} 
\doublespacing
\usepackage{indentfirst}
\setlength{\parindent}{25pt}
\usepackage{bbm}
\usepackage{multicol}
\usepackage{makecell}
\usepackage{booktabs}
\newtheorem{prop}{Proposition} 
\newtheorem{assumption}{Assumption} 
\newtheorem{implication}{Implication} 
\usepackage{longtable}

\setcounter{totalnumber}{8}

\title{Winter weather on exam dates and matriculation for a prestigious university in Japan}
\author{
  Mizuhiro Suzuki\thanks{
    Contact: mizuhiro.suzuki@gmail.com.
    The author would like to thank Laura Schechter for her numerous feedback and support.
    I am also grateful to Daiji Kawaguchi for helpful comments and suggestions.
    Any errors are the sole responsibility of the author.
    The author declares that he has no relevant or material financial interests that relate to the research described in this paper.
  }
}
\date{\today}

\begin{document}
  
\maketitle
\begin{abstract}
  \singlespacing
    \noindent 
    Whereas there are growing numbers of studies on how heat affects cognitive performance at high-stakes settings, the effect of cold weather has been less studied.
    To fill this gap, I analyze the effects of winter weather on exam dates on performance at the centralized exam for university admission in Japan.
    Since students take the exam in their own prefectures, they are exposed to different weather conditions on the exam dates. 
    I find that low temperature and snow cumulated on the ground reduce matriculation shares for the most prestigious university in Japan, the University of Tokyo.
    This result suggests that the winter weather on exam dates worsens exam performance. 

  \medskip
  \vspace{1cm}
  \noindent Keywords: Temperature, Education, University matriculation, Cognitive performance
  \vspace{1cm}
  JEL Codes: I21, Q54, J24
\end{abstract}

\newpage

\section{Introduction}

University admission is an important step in life and can have consequences in later-life outcomes such as job occupation and earnings.\footnote{
  \citet{Hout2012} reviews the returns to college education in the context of the United States.
}
For admission selection, standardized exams have been widely used in many countries.\footnote{
  Examples include the United States, the United Kingdom, Brazil, China, South Korea, and Japan.
}
Given the importance, the results of the exams are expected to evaluate the exam takers' ability and knowledge correctly and are not supposed to be affected by seemingly irrelevant factors.
However, recent studies have shown that exam performance is sensitive to external factors.
One such factor is temperature:
heat on exam days has been shown to affect exam results in high-stakes settings \citep{Park2020a, GraffZivin2020, Melo2019}.
What has been less studied in the literature is the impact of \textit{low} temperature.\footnote{
  One might think that, considering the ongoing global \textit{warming}, the effect of low temperature is becoming less important.
  However, future climate change can cause sea-ice loss, which could result in the cold winters with severely low temperatures \citep{Kretschmer2016, Kim2014}.
  Hence, it is important to understand how low temperature can affect our economies and lives.
}

To fill this gap in the literature, this study analyzes the effect of winter weather on exam outcomes in Japan.
For admission, most public universities in Japan require high school seniors and graduates to take a centralized test, called National Center Test for University Admission (Center Test henceforth).
Students take the exam on the same dates in mid-January in their prefectures.\footnote{
  There are 47 prefectures in Japan.
}
This creates a variation in weather to which they are exposed on the exam dates.

Unlike the settings in the previous studies looking at the heat effects on exam scores, the score data of individual exam takers or even the aggregated exam score information are not easily accessible to researchers.
To overcome this limitation, I use the weather on the dates of the Center Test and investigate its effect on matriculation for the University of Tokyo (UTokyo henceforth).
UTokyo is a national university known to be the most prestigious in Japan.\footnote{
  According to \citet{usnews}, UTokyo is the highest-ranked university in Japan (73rd ranked in the world ranking).
  The second highest-ranked university in Japan is Kyoto University, which is 125th ranked in the world ranking.
}
This allows me to attribute a reduction in matriculation shares to the worse exam performance, instead of the better performance, which could induce students to choose better universities.

The econometric analyses show that low temperature and snow cumulated on the ground on exam dates reduce the matriculation share.
The estimates are not only statistically but also economically significant.
Changing the temperature on the exam dates from 3-6 $^o$C to below 0 $^o$C decreases the matriculation share by 0.18\%, which amounts to 9\% of the average or 22\% of the median matriculation shares.
%The results provide implications for effective measures to be taken to mitigate the negative impact of winter weather.
%These include preparing waiting rooms for exam takers to warm up to equalize their pre-exam conditions or adjusting the importance of the scores at the Center Test in the admissions process.

This study mainly contributes to the recent but growing literature on the effects of climate factors on the outcome of high-stakes exams \citep{Park2020b, Cho2017}.
I also contribute to the literature on the economic impact of low temperature.
Whereas it has been shown with controlled experiments that cold stresses negatively affect such things as cognitive ability and memory, as reviewed in \citet{Taylor2016}, evidence based on data in the field is scarce \citep{Burke2015, Stevens2017}.
I use prefecture-level information to investigate how low temperature affects cognitive performance in a high-stakes exam.

This study proceeds as follows.
Section \ref{sec:data} describes the data. 
I introduce my empirical strategy in Section \ref{sec:empirical_strategy}.
In Section \ref{sec:results}, I present the empirical results, and Section \ref{sec:conclusion} concludes.
  
\section{Data}\label{sec:data}

I use two sets of data: weather and matriculation for UTokyo.
For weather information, I obtain data from \Citeauthor{jma}.
Their website maintains the past weather across Japan.
Based on hourly weather on exam dates in prefecture capitals, I use average weather between 6 AM and 6 PM.
Prefecture capitals tend to be the most populated in each prefecture, and highly-ranked high schools from which many students go to UTokyo are likely to be located in prefecture capitals.

For matriculation information for UTokyo, I use University Basic Information by \Citeauthor{univ_info}.
This data contains information from 2012 to 2020 on the number of matriculated students for each university by prefectures at which their high schools are located.
I calculate the matriculation shares of each prefecture as the ratio of the number of matriculated students from a prefecture to the number of all matriculated students.
The summary statistics are provided in Table \ref{tab:sum_stat}.

\begin{table}[!htbp]
  \centering
  \caption{Summary Statistics}
  \resizebox{0.8\linewidth}{!}{
  
\begin{tabular}[t]{lllllll}
\toprule
\multicolumn{1}{c}{ } & \multicolumn{1}{c}{N} & \multicolumn{1}{c}{Mean} & \multicolumn{1}{c}{SD} & \multicolumn{1}{c}{Median} & \multicolumn{1}{c}{Min} & \multicolumn{1}{c}{Max} \\
\cmidrule(l{3pt}r{3pt}){2-2} \cmidrule(l{3pt}r{3pt}){3-3} \cmidrule(l{3pt}r{3pt}){4-4} \cmidrule(l{3pt}r{3pt}){5-5} \cmidrule(l{3pt}r{3pt}){6-6} \cmidrule(l{3pt}r{3pt}){7-7}
Matriculation share (\%) & 517 & 2.08 & 5.22 & 0.76 & 0.06 & 37.35\\
Temperature (\degree C) & 423 & NA & NA & NA & NA & NA\\
Hourly precipitation (mm) & 423 & NA & NA & NA & NA & NA\\
Hourly snowfall (cm) & 423 & NA & NA & NA & NA & NA\\
Cumulated snow (cm) & 423 & NA & NA & NA & NA & NA\\
\bottomrule
\end{tabular}

  }
  \label{tab:sum_stat}
  \footnotesize
  \begin{tablenotes}
    \item 
      Notes:
      Matriculation share (\%) is the share of newly matriculated students to the University of Tokyo from each prefecture.
      The matriculation share is calculated based on the ratio of the number of students from high schools in the prefecture to the total number of matriculation.
      I use weather variables in prefecture capitals between 6 AM and 6 PM on days of the Center Test.
  \end{tablenotes}
\end{table}

\section{Empirical Strategy}\label{sec:empirical_strategy}

I analyze the effect of temperature and other weather variables on exam dates on the matriculation share for UTokyo.
The reason why I use the matriculation share for UTokyo is as follows:
consider a matriculation share for a middle-ranked university.
Suppose also that low temperature has a negative impact on exam scores.
Then, 
(i) students in the prefecture who would have gotten into the middle-ranked university if the temperature were higher may end up in a lower-ranked university instead, but
(ii) students in the prefecture who would have gotten into a higher-ranked university may end up in the middle-ranked university instead.
Hence, the matriculation share of the prefecture for the middle-ranked university may not be affected by low temperature.
However, the second effect is unlikely to be large for UTokyo since it is known to be the best university in Japan.\footnote{
  \citet{Kawaguchi2008} describes that ``The University of Tokyo has occupied the top position of the single-peaked university hierarchy since its establishment in 1877.''
}
Therefore, I rule out the second possibility and attribute a reduction in matriculation shares to the worse performance at the Center Test.\footnote{
  Appendix \ref{sec:background} details the admission system for UTokyo.
}

In the analyses, the outcome variable, $Y_{jt}$, is the matriculation share (\%) for UTokyo of a prefecture $j$ in year $t$.
The main right-hand side variable is temperature on the exam dates.
To allow for non-linear effects of temperature on the outcome, I use the 3-$^o$C temperature intervals.
%\footnote{
%  The results with a linear specification are shown in Appendix \ref{sec:appendix_table} and will be discussed below.
%}
Let $T_{jt}^k$ represent the $k$'th temperature bin.
I also include other weather variables such as rainfall, snowfall, and cumulated snow on the ground on exam dates as a vector $X_{jt}$.
As fixed effects, prefecture fixed effects ($\mu_j$) and year fixed effects ($\tau_t$) are used.
Finally, the error term is represented as $\epsilon_{jt}$.

Given the above notation, I run the following regression equation:
\begin{equation*}
  Y_{jt} = \sum_k T_{jt}^k + X_{jt}' \beta + \mu_j + \tau_t + \epsilon_{jt}.
\end{equation*}
I exploit the variation of temperature on exam dates, which I consider exogenous after controlling for prefecture-specific time-invariant factors and time-varying aggregate shocks through prefecture and year fixed effects.
The standard errors are clustered at the prefecture level.
As Figure \ref{fig:temperature_diff} shows, each prefecture experiences substantial variations in temperature on exam dates.\footnote{
  Figures for other weather variables are provided in Appendix \ref{sec:appendix_figure}.
}$^,$\footnote{
  The figure shows that temperature is low for most prefectures in some years (eg. 2017) and high in other years (eg. 2019).
  This can bring up a concern that temperature \textit{deviations} from the prefecture average have limited variations within each year.
  If this were the case, most variations for estimation would be absorbed by prefecture fixed effects and year fixed effects.
  Figure \ref{fig:temperature_diff_by_year} shows that, even within each year, the temperature deviations have large variations across prefectures.
}

\begin{figure}[H]
  \centering
  \caption{Average temperature (\degree C) on two exam dates in each prefecture each year}
  \includegraphics[width = 0.9\textwidth]{../Output/images/temperature_diff.pdf}
  \label{fig:temperature_diff}
  \footnotesize
  \begin{tablenotes}
    \item Notes:
      Temperature in prefecture capitals between 6 AM and 6 PM on days of the Center Test is used.
      The average across two test days in each year is shown in the figure.
      The prefectures are ordered by the mean temperature across years (lowest to left and highest to right).
  \end{tablenotes}
\end{figure}

\section{Results}\label{sec:results}

The main empirical results are provided in Table \ref{tab:main_reg}.
Without other weather variables, the estimates in column (1) show a negative and statistically significant impact of cold weather.
Since the mean matriculation share is $2.08$\%, a decrease in the temperature from the reference bin (between 3 and 6 $^o$C) to the lowest temperature bin (below 0 $^o$C) results in the lower matriculation share by 9\%.
Moreover, due to the stark skewness in the distribution of matriculation shares, the median matriculation share is $0.79$\%.
Therefore, the same temperature change decreases the matriculation share by 22\% of the median value.

In column (2), I add rainfall and snowfall in the regressions.
While temperature coefficients are barely affected, I find that rainfall and snowfall on the exam dates do not impact the matriculation shares.
In column (3), instead of snowfall on the exam dates, I use cumulated snow on the ground and find its negative impact on the outcome. 
To capture the effect of the existence of a certain amount of snow on the ground, column (4) allows a non-linear effect of cumulated snow.
The result shows that snow on the ground cumulated more than 10 cm ($\approx$ 4 inches) decreases the matriculation share by 0.11 percentage points.
These results suggest that not only temperature but also snow on the ground on the exam dates affect exam performance.

Two points are worth noting.
First, Table \ref{tab:linear_reg} shows the results with linear specifications.
While the point estimates for temperature are positive, which means matriculation shares are higher if an exam is held on warmer days, they are statistically insignificant. 
This could be due to the plateau effect of temperature above 6 $^o$C in Table \ref{tab:main_reg}.
Although for a quite different temperature range, similar plateau effects are found in \citet{Park2020a}.
Secondly, following the falsification test in \citet{Cho2017}, I check if weather variables one year after the actual the Center Test affect the matriculation shares.
The results in Table \ref{tab:reg_placebo_exam} show not only that the weather one year after the Center Test has insignificant impacts on matriculation shares but also including the weather variables one year after the Center Test barely changes the coefficients on contemporaneous weather variables.
This indicates that the results in \ref{tab:main_reg} capture the effect of transitory shocks on the exam dates.
In Table \ref{tab:reg_pre10}, I include the average weather variables for 10 days before the exams and run regressions.
The coefficients are statistically insignificant, which suggests that the effects of weather on preparation for the exam is secondary and the weather on the exam dates has the primary impact on exam scores.

\begin{table}[H]
  \center
  \caption{Regression: Matriculation share (\%) and weather on exam dates}
  \footnotesize
  
% Table created by stargazer v.5.2.2 by Marek Hlavac, Harvard University. E-mail: hlavac at fas.harvard.edu
% Date and time: Thu, Apr 22, 2021 - 23:18:23
\begin{tabular}{@{\extracolsep{5pt}}lcccccccc} 
\\[-1.8ex]\hline 
\hline \\[-1.8ex] 
 & \multicolumn{8}{c}{\textit{Dependent variable:}} \\ 
\cline{2-9} 
\\[-1.8ex] & \multicolumn{8}{c}{Matriculation share (\%)} \\ 
\\[-1.8ex] & (1) & (2) & (3) & (4) & (5) & (6) & (7) & (8)\\ 
\hline \\[-1.8ex] 
 Temperature (\degree C) & 0.02 &  & 0.02 &  & 0.02 &  & 0.02 &  \\ 
  & (0.02) &  & (0.02) &  & (0.02) &  & (0.02) &  \\ 
  & & & & & & & & \\ 
 Temperature (\degree C) $\le$ 0 &  & $-$0.18$^{**}$ &  & $-$0.19$^{**}$ &  & $-$0.16$^{**}$ &  & $-$0.15$^{*}$ \\ 
  &  & (0.08) &  & (0.08) &  & (0.08) &  & (0.08) \\ 
  & & & & & & & & \\ 
 Temperature (\degree C) $>$ 0, $\le$ 3 &  & $-$0.08$^{*}$ &  & $-$0.08$^{*}$ &  & $-$0.08$^{*}$ &  & $-$0.08$^{*}$ \\ 
  &  & (0.04) &  & (0.04) &  & (0.04) &  & (0.04) \\ 
  & & & & & & & & \\ 
 Temperature (\degree C) $>$ 6, $\le$ 9 &  & 0.13 &  & 0.13 &  & 0.13 &  & 0.13 \\ 
  &  & (0.10) &  & (0.09) &  & (0.09) &  & (0.09) \\ 
  & & & & & & & & \\ 
 Temperature (\degree C) $>$ 9 &  & 0.14 &  & 0.14 &  & 0.15 &  & 0.15 \\ 
  &  & (0.12) &  & (0.12) &  & (0.12) &  & (0.12) \\ 
  & & & & & & & & \\ 
 Rainfall (mm) &  &  & $-$0.04 & $-$0.02 & 0.04 & 0.06 & 0.05 & 0.06 \\ 
  &  &  & (0.10) & (0.09) & (0.08) & (0.08) & (0.09) & (0.08) \\ 
  & & & & & & & & \\ 
 Snowfall (m) &  &  & 7.77 & 7.75 &  &  &  &  \\ 
  &  &  & (10.91) & (9.95) &  &  &  &  \\ 
  & & & & & & & & \\ 
 Cumulated snow (m) &  &  &  &  & $-$0.18$^{**}$ & $-$0.19$^{**}$ &  &  \\ 
  &  &  &  &  & (0.09) & (0.09) &  &  \\ 
  & & & & & & & & \\ 
 Cumulated snow $>$ .10 m &  &  &  &  &  &  & $-$0.11$^{**}$ & $-$0.11$^{***}$ \\ 
  &  &  &  &  &  &  & (0.04) & (0.04) \\ 
  & & & & & & & & \\ 
\hline \\[-1.8ex] 
Prefecture FE & Yes & Yes & Yes & Yes & Yes & Yes & Yes & Yes \\ 
Year FE & Yes & Yes & Yes & Yes & Yes & Yes & Yes & Yes \\ 
Observations & 423 & 423 & 423 & 423 & 423 & 423 & 423 & 423 \\ 
\hline 
\hline \\[-1.8ex] 
\textit{Note:}  & \multicolumn{8}{r}{$^{*}$p$<$0.1; $^{**}$p$<$0.05; $^{***}$p$<$0.01} \\ 
\end{tabular} 

  \label{tab:main_reg}
  \small
  \begin{tablenotes}
    \item
      The outcome variable is the matriculation shares (\%) for the University of Tokyo from each prefecture.
      The matriculation share is calculated based on the ratio of the number of students from high schools in the prefecture to the total number of matriculation.
      I use weather variables in prefecture capitals between 6 AM and 6 PM on days of the Center Test.
      Temperature intervals do not contain the left ends.
      Standard errors are clustered at the prefecture level.
  \end{tablenotes}
\end{table}

\section{Conclusion}\label{sec:conclusion}

This paper studies the impact of winter weather on cognitive performance on a high-stakes exam in Japan.
I use the matriculation share for the most prestigious university in Japan, the University of Tokyo, as an outcome.
Most of the existing studies in the literature have focused on the effect of heat on cognitive performance, while the impact of cold weather remains largely unknown.
This study fills the gap by identifying the causal impact of winter weather on exam results.

The results show both economically and statistically significant impacts of low temperature and cumulated snow on the ground on the exam dates.
Changing the temperature from 3-6 $^o$C to below 0 $^o$C decreases the matriculation share of a prefecture by 0.18 percentage points, which is 9\% of the mean matriculation share, or 22\% of the median matriculation share.
Moreover, having more than 10cm of snow on the ground decreases the share by 0.11 percentage points.

The results in this study have implications for the admissions process of universities.
Recognizing the limitation of measuring students' cognitive ability through an exam affected by external factors such as weather, universities may use other measures to select students.
Examples include extracurricular activities and personal statements, which are actively utilized in other countries such as the United States.
Also, universities may adjust the weights for the Center Test and the second-stage exam.
Since the second-stage exams are held at each university, this rules out the possibility that exam takers at different places are affected differently by their conditions.
Given the importance of exam results for later life outcomes of students, it is desirable that universities take measures to mitigate the effect of seemingly irrelevant factors on admission.

\clearpage
\bibliographystyle{apalike}
\bibliography{todai}

\appendix

\section{Background}\label{sec:background}

In Japan, for admission to public universities, students are usually required to take two exams.
The first-stage test is the Center Test.
Students take an exam on two consecutive days (Saturday and Sunday) in mid-January.
Most public universities, including UTokyo, require students to take exams on social science, Japanese, and English on the first day and mathematics and natural science on the second day.
The exams start at 9:30 AM on both days and end at 6:10 PM on the first day and 5:50 PM on the second day.
Questions are multiple-choice and common for all exam takers across Japan.
High-school seniors take the test at designated sites in the exam area\footnote{ \label{footnote:prefecture}
  Typically, an exam area coincides with a prefecture (i.e. there is one exam area in one prefecture).
  Large prefectures such as Hokkaido and Okinawa have several exam areas.
  Also, several municipalities in a prefecture can be sometimes considered to be a part of a neighboring prefecture.
  For instance, students in a couple of municipalities in Saitama prefecture could take the Center Test in Tokyo.
} where their high schools are located.
On the other hand, high-school graduates are assigned to test sites in the exam area where the homes used in their applications are located.\footnote{
  \citet{Watanabe2013} provides detailed descriptions of the administrations of the Center Test.
}

Since January is the middle of winter in Japan, heavy snow can affect public transportation on exam dates.
This sometimes makes it difficult for exam takers to arrive at exam sites on time.
For the affected students, the following accommodations can be provided:
(i) delaying the start of exams by a few hours for all exam takers at a test site,
(ii) allowing the affected students to take exams in a separate room, and
(iii) allowing the affected students to take make-up exams a week later.
These accommodations can be given to students who could not take exams for other reasons such as sickness or accidents.

After the Center Test, applicants for public universities typically take the second-stage exam.
This exam is made by each university, and students physically go to the universities and take the exam.
Therefore, every applicant for a university takes the second-stage exam at the same location on the same days.
UTokyo has the second-stage exam at the end of February for two days.\footnote{
  Until 2015, UTokyo had another second-stage exam in March, and a small number of exam takers got admitted through the exam (around 3\% of all admitted students).
  In 2016, UTokyo abolished the March exam and instead started to admit the small number of students recommended by their high schools based on interviews and scores at the Center Test (around 2\% of all admitted students).
  Since the majority of students are admitted through the second-stage exam in February, below I focus on the admissions process for those students. 
}

In admission for UTokyo, scores at the Center Test are used for two purposes.
First, the scores are used for eligibility to take the second-stage exam.
Thresholds varying across years and majors are set, and if scores at the Center Test are below the thresholds, the students cannot take the second-stage exam.
Secondly, the scores at the Center Test are taken into consideration for admission.
The final score of a student is calculated by the weighted sum of exam scores at the first- and second-stage exams.\footnote{
  In 2020, the weights for the Center Test and the second-stage exam are 20\% and 80\%.
  The Faculty of Medicine additionally conducts interviews on the next day of the second-stage exam to screen applicants based on community skills and personal maturity.
}

%In summary, results at the Center Test are crucial for applicants for UTokyo.
%This study will investigate how external factors such as temperature and weather on the dates of the Center Test affect matriculation for UTokyo.


\setcounter{figure}{0}
\setcounter{table}{0}
\renewcommand\thefigure{\Alph{section}.\arabic{figure}}
\renewcommand\thetable{\Alph{section}.\arabic{table}}
  
\section{Appendix tables}\label{sec:appendix_table}

\begin{table}[H]
  \center
  \caption{Regression: Matriculation share (\%) and weather on exam dates (linear specification)}
  \small
  
% Table created by stargazer v.5.2.2 by Marek Hlavac, Harvard University. E-mail: hlavac at fas.harvard.edu
% Date and time: Thu, May 13, 2021 - 14:05:22
\begin{tabular}{@{\extracolsep{5pt}}lcccc} 
\\[-1.8ex]\hline 
\hline \\[-1.8ex] 
 & \multicolumn{4}{c}{\textit{Dependent variable:}} \\ 
\cline{2-5} 
\\[-1.8ex] & \multicolumn{4}{c}{Matriculation share (\%)} \\ 
\\[-1.8ex] & (1) & (2) & (3) & (4)\\ 
\hline \\[-1.8ex] 
 Temperature (\degree C) & 0.02 & 0.02 & 0.02 & 0.02 \\ 
  & (0.02) & (0.02) & (0.02) & (0.02) \\ 
  & & & & \\ 
 Rainfall (mm) &  & $-$0.04 & 0.04 & 0.05 \\ 
  &  & (0.10) & (0.08) & (0.09) \\ 
  & & & & \\ 
 Snowfall (cm) &  & 0.08 &  &  \\ 
  &  & (0.11) &  &  \\ 
  & & & & \\ 
 Cumulated snow (cm) &  &  & $-$0.002$^{**}$ &  \\ 
  &  &  & (0.001) &  \\ 
  & & & & \\ 
 Cumulated snow $>$ 10 cm &  &  &  & $-$0.11$^{**}$ \\ 
  &  &  &  & (0.04) \\ 
  & & & & \\ 
\hline \\[-1.8ex] 
Prefecture FE & Yes & Yes & Yes & Yes \\ 
Year FE & Yes & Yes & Yes & Yes \\ 
Observations & 423 & 423 & 423 & 423 \\ 
\hline 
\hline \\[-1.8ex] 
\textit{Note:}  & \multicolumn{4}{r}{$^{*}$p$<$0.1; $^{**}$p$<$0.05; $^{***}$p$<$0.01} \\ 
\end{tabular} 

  \label{tab:linear_reg}
  \small
  \begin{tablenotes}
    \item
      The outcome variable is the matriculation shares (\%) for the University of Tokyo from each prefecture.
      The matriculation share is calculated based on the ratio of the number of students from high schools in the prefecture to the total number of matriculation.
      I use weather variables in prefecture capitals between 6 AM and 6 PM on days of the Center Test.
      Standard errors are clustered at the prefecture level.
  \end{tablenotes}
\end{table}

\begin{table}[H]
  \center
  \caption{Falsification test: Matriculation share (\%) and weather on exam dates and one year after}
  \scriptsize
  
% Table created by stargazer v.5.2.2 by Marek Hlavac, Harvard University. E-mail: hlavac at fas.harvard.edu
% Date and time: Mon, Apr 05, 2021 - 14:31:09
\begin{tabular}{@{\extracolsep{5pt}}lcccc} 
\\[-1.8ex]\hline 
\hline \\[-1.8ex] 
 & \multicolumn{4}{c}{\textit{Dependent variable:}} \\ 
\cline{2-5} 
\\[-1.8ex] & \multicolumn{4}{c}{Matriculation share (\%)} \\ 
\\[-1.8ex] & (1) & (2) & (3) & (4)\\ 
\hline \\[-1.8ex] 
 Temperature ($t$) $\le$ 0 &  &  & $-$0.17$^{**}$ & $-$0.14$^{*}$ \\ 
  &  &  & (0.08) & (0.08) \\ 
  & & & & \\ 
 Temperature ($t$) $>$ 0, $\le$ 3 &  &  & $-$0.07$^{*}$ & $-$0.07 \\ 
  &  &  & (0.04) & (0.04) \\ 
  & & & & \\ 
 Temperature ($t$) $>$ 6, $\le$ 9 &  &  & 0.12 & 0.13 \\ 
  &  &  & (0.09) & (0.09) \\ 
  & & & & \\ 
 Temperature ($t$) $>$ 9 &  &  & 0.14 & 0.15 \\ 
  &  &  & (0.11) & (0.11) \\ 
  & & & & \\ 
 Rainfall (mm) ($t$) &  &  & $-$0.07 & 0.04 \\ 
  &  &  & (0.09) & (0.08) \\ 
  & & & & \\ 
 Snowfall (m) ($t$) &  &  & 11.31 &  \\ 
  &  &  & (9.22) &  \\ 
  & & & & \\ 
 Cumulated snow (m) ($t$) &  &  &  & $-$0.22$^{**}$ \\ 
  &  &  &  & (0.09) \\ 
  & & & & \\ 
 Temperature ($t + 1$) $\le$ 0 & 0.02 & $-$0.05 & $-$0.01 & $-$0.03 \\ 
  & (0.12) & (0.11) & (0.11) & (0.10) \\ 
  & & & & \\ 
 Temperature ($t + 1$) $>$ 0, $\le$ 3 & $-$0.03 & $-$0.05 & $-$0.04 & $-$0.05 \\ 
  & (0.05) & (0.05) & (0.05) & (0.05) \\ 
  & & & & \\ 
 Temperature ($t + 1$) $>$ 6, $\le$ 9 & 0.04 & 0.04 & 0.05 & 0.04 \\ 
  & (0.05) & (0.05) & (0.05) & (0.05) \\ 
  & & & & \\ 
 Temperature ($t + 1$) $>$ 9 & $-$0.09 & $-$0.09 & $-$0.05 & $-$0.06 \\ 
  & (0.08) & (0.08) & (0.07) & (0.07) \\ 
  & & & & \\ 
 Rainfall (mm) ($t + 1$) & $-$0.01 & $-$0.03 & $-$0.01 & $-$0.03 \\ 
  & (0.05) & (0.05) & (0.05) & (0.05) \\ 
  & & & & \\ 
 Snowfall (m) ($t + 1$) & $-$9.10 &  & $-$4.94 &  \\ 
  & (12.08) &  & (11.90) &  \\ 
  & & & & \\ 
 Cumulated snow (m) ($t + 1$) &  & 0.13 &  & 0.11 \\ 
  &  & (0.12) &  & (0.12) \\ 
  & & & & \\ 
\hline \\[-1.8ex] 
Prefecture FE & Yes & Yes & Yes & Yes \\ 
Year FE & Yes & Yes & Yes & Yes \\ 
Observations & 423 & 423 & 423 & 423 \\ 
\hline 
\hline \\[-1.8ex] 
\textit{Note:}  & \multicolumn{4}{r}{$^{*}$p$<$0.1; $^{**}$p$<$0.05; $^{***}$p$<$0.01} \\ 
\end{tabular} 

  \label{tab:reg_placebo_exam}
  \scriptsize
  \begin{tablenotes}
    \item
      The outcome variable is the matriculation shares (\%) for the University of Tokyo from each prefecture.
      The matriculation share is calculated based on the ratio of the number of students from high schools in the prefecture to the total number of matriculation.
      I use weather variables in prefecture capitals between 6 AM and 6 PM on days of the Center Test.
      I represent the contemporaneous weather variables with ``($t$)'' and the weather one year after the Center Test with ``($t + 1$).''
      Temperature intervals do not contain the left ends.
      Standard errors are clustered at the prefecture level.
  \end{tablenotes}
\end{table}

\begin{table}[H]
  \center
  \caption{Regression: Matriculation share (\%) and average weather for 10 days before exam}
  \scriptsize
  
% Table created by stargazer v.5.2.3 by Marek Hlavac, Social Policy Institute. E-mail: marek.hlavac at gmail.com
% Date and time: Sat, Mar 02, 2024 - 12:41:22
\begin{tabular}{@{\extracolsep{5pt}}lcccc} 
\\[-1.8ex]\hline 
\hline \\[-1.8ex] 
 & \multicolumn{4}{c}{\textit{Dependent variable:}} \\ 
\cline{2-5} 
\\[-1.8ex] & \multicolumn{4}{c}{Matriculation share (\%)} \\ 
\\[-1.8ex] & (1) & (2) & (3) & (4)\\ 
\hline \\[-1.8ex] 
 Temperature (\degree C) $\le$ 0 (exam dates) &  &  & $-$0.18$^{*}$ & $-$0.18$^{**}$ \\ 
  &  &  & (0.09) & (0.08) \\ 
  & & & & \\ 
 Temperature (\degree C) 0-3 (exam dates) &  &  & $-$0.11$^{**}$ & $-$0.11$^{**}$ \\ 
  &  &  & (0.05) & (0.05) \\ 
  & & & & \\ 
 Temperature (\degree C) 6-9 (exam dates) &  &  & 0.12 & 0.12 \\ 
  &  &  & (0.10) & (0.10) \\ 
  & & & & \\ 
 Temperature (\degree C) $>$ 9 (exam dates) &  &  & 0.12 & 0.13 \\ 
  &  &  & (0.12) & (0.13) \\ 
  & & & & \\ 
 Rainfall (mm) (exam dates) &  &  & $-$0.02 & 0.06 \\ 
  &  &  & (0.10) & (0.08) \\ 
  & & & & \\ 
 Snowfall (cm) (exam dates) &  &  & 0.12 &  \\ 
  &  &  & (0.10) &  \\ 
  & & & & \\ 
 Cumulated snow (cm) (exam dates) &  &  &  & $-$0.002$^{*}$ \\ 
  &  &  &  & (0.001) \\ 
  & & & & \\ 
 Temperature (\degree C) $\le$ 0 (previous 10 days) & 0.03 & $-$0.002 & 0.10 & 0.07 \\ 
  & (0.08) & (0.09) & (0.10) & (0.10) \\ 
  & & & & \\ 
 Temperature (\degree C) 0-3 (previous 10 days) & 0.05 & 0.04 & 0.10 & 0.09 \\ 
  & (0.05) & (0.05) & (0.06) & (0.06) \\ 
  & & & & \\ 
 Temperature (\degree C) 6-9 (previous 10 days) & 0.09 & 0.09 & 0.07 & 0.07 \\ 
  & (0.06) & (0.06) & (0.06) & (0.06) \\ 
  & & & & \\ 
 Temperature (\degree C) $>$ 9 (previous 10 days) & 0.01 & 0.01 & $-$0.005 & $-$0.01 \\ 
  & (0.11) & (0.11) & (0.12) & (0.11) \\ 
  & & & & \\ 
 Rainfall (mm) (previous 10 days) & $-$0.003 & $-$0.01 & $-$0.003 & $-$0.004 \\ 
  & (0.01) & (0.01) & (0.01) & (0.01) \\ 
  & & & & \\ 
 Snowfall (cm) (previous 10 days) & $-$0.02$^{*}$ &  & $-$0.02 &  \\ 
  & (0.01) &  & (0.01) &  \\ 
  & & & & \\ 
 Cumulated snow (cm) (previous 10 days) &  & $-$0.002$^{*}$ &  & 0.001 \\ 
  &  & (0.001) &  & (0.002) \\ 
  & & & & \\ 
\hline \\[-1.8ex] 
Prefecture FE & Yes & Yes & Yes & Yes \\ 
Year FE & Yes & Yes & Yes & Yes \\ 
Observations & 423 & 423 & 423 & 423 \\ 
\hline 
\hline \\[-1.8ex] 
\textit{Note:}  & \multicolumn{4}{r}{$^{*}$p$<$0.1; $^{**}$p$<$0.05; $^{***}$p$<$0.01} \\ 
\end{tabular} 

  \label{tab:reg_pre10}
  \scriptsize
  \begin{tablenotes}
    \item
      The outcome variable is the matriculation shares (\%) for the University of Tokyo from each prefecture.
      The matriculation share is calculated based on the ratio of the number of students from high schools in the prefecture to the total number of matriculation.
      For weather variables with "(exam dates)", I use weather variables in prefecture capitals between 6 AM and 6 PM on days of the Center Test.
      For weather variables with "(previous 10 days)", I use their averages in prefecture capitals between 10 days before and 1 day before the first day of the Center Test.
      Temperature intervals do not contain the left ends.
      Standard errors are clustered at the prefecture level.
  \end{tablenotes}
\end{table}


\setcounter{figure}{0}
\setcounter{table}{0}
\renewcommand\thefigure{\Alph{section}.\arabic{figure}}
\renewcommand\thetable{\Alph{section}.\arabic{table}}
  
\section{Appendix figures}\label{sec:appendix_figure}

\begin{figure}[H]
  \centering
  \caption{Average hourly rainfall (m) on two exam dates in each prefecture each year}
  \includegraphics[width = 0.9\textwidth]{../Output/images/rainfall_diff.pdf}
  \label{fig:rainfall_diff}
  \footnotesize
  \begin{tablenotes}
    \item Notes:
      Rainfall (mm) in prefecture capitals between 6 AM and 6 PM on days of the Center Test is used.
      Average across two test days in each year is shown in the figure.
      The prefectures are ordered by the mean rainfall across years (lowest to left and highest to right).
  \end{tablenotes}
\end{figure}

\begin{figure}[H]
  \centering
  \caption{Average hourly snowfall (m) on two exam dates in each prefecture each year}
  \includegraphics[width = 0.9\textwidth]{../Output/images/snowfall_diff.pdf}
  \label{fig:snowfall_diff}
  \footnotesize
  \begin{tablenotes}
    \item Notes:
      Snowfall (m) in prefecture capitals between 6 AM and 6 PM on days of the Center Test is used.
      Average across two test days in each year is shown in the figure.
      The prefectures are ordered by the mean snowfall across years (lowest to left and highest to right).
  \end{tablenotes}
\end{figure}

\begin{figure}[H]
  \centering
  \caption{Average cumulated snow (m) on two exam dates in each prefecture each year}
  \includegraphics[width = 0.9\textwidth]{../Output/images/cum_snow_diff.pdf}
  \label{fig:cum_snow_diff}
  \footnotesize
  \begin{tablenotes}
    \item Notes:
      Cumulated snow (m) in prefecture capitals between 6 AM and 6 PM on days of the Center Test is used.
      Average across two test days in each year is shown in the figure.
      The prefectures are ordered by the mean cumulated snow across years (lowest to left and highest to right).
  \end{tablenotes}
\end{figure}

\begin{figure}[H]
  \centering
  \caption{Deviation of temperatures from within-prefecture averages}
  \includegraphics[width = 0.9\textwidth]{../Output/images/temperature_diff_by_year.pdf}
  \label{fig:temperature_diff_by_year}
  \footnotesize
  \begin{tablenotes}
    \item Notes:
      Temperature in prefecture capitals between 6 AM and 6 PM on days of the Center Test is used.
      Deviations of average temperatures across two test days from their within-prefecture means are shown in the figure by year.
      In each panel, red dots are the deviations in the year shown above and gray dots are the deviations in the other years.
      The prefectures are ordered by the mean temperature across years (lowest to left and highest to right).
  \end{tablenotes}
\end{figure}

  
\end{document}



