\documentclass[9pt, pdfmx,hiresbb]{beamer}
\usepackage{natbib}
\usepackage{bbm}
\usepackage[utf8]{inputenc}
\usepackage{amsmath}
\usepackage{amsfonts}
\usepackage{amssymb}
\usepackage{amsthm}
\usepackage{textcomp}
\usepackage{gensymb}
\usepackage{appendixnumberbeamer}
\usepackage{siunitx}
\usepackage{lipsum}
\usepackage{float}
\usepackage{pdflscape}
\usepackage{indentfirst}
\usepackage{longtable}
\usepackage{multicol}
\usepackage{makecell}
\usepackage{booktabs}
\usepackage{threeparttable}
\usepackage{graphicx}
\usepackage{setspace} 

\usepackage{color}  % https://dkumor.com/posts/technical/2018/08/15/causal-tikz/
\usepackage{tikz}
\usetikzlibrary{shapes,decorations,arrows,calc,arrows.meta,fit,positioning}
\tikzset{
    -Latex,auto,node distance =1 cm and 1 cm,semithick,
    state/.style ={ellipse, draw, minimum width = 0.7 cm},
    point/.style = {circle, draw, inner sep=0.04cm,fill,node contents={}},
    bidirected/.style={Latex-Latex,dashed},
    el/.style = {inner sep=2pt, align=left, sloped}
}

\newtheorem{prop}{Proposition} 
\setbeamertemplate{thorems}[numbered]

\setbeamerfont{page number in head/foot}{size=\scriptsize}
\setbeamertemplate{footline}[frame number]

\newcommand{\indep}{\rotatebox[origin=c]{90}{$\models$}}

\newcommand\Wider[2][3em]{
\makebox[\linewidth][c]{
  \begin{minipage}{\dimexpr\textwidth+#1\relax}
  \raggedright#2
  \end{minipage}
  }
}

%\usetheme{CambridgeUS}
\setbeamertemplate{navigation symbols}{}
\title[Temperature and Exam]{Temperature and Bureaucracy: \\ Empirical Analysis on Auditing Reports in Brazil}
\author[Suzuki]{Mizuhiro Suzuki}
\date{May 13th, 2021}

\begin{document}
\begin{frame}
\titlepage
\end{frame}

\section{Background}
\begin{frame}\frametitle{Roadmap}
  \begin{itemize}
    \item Background
    \item Data
    \item Empirical strategy
    \item Results
    \item (Some more results not in the paper for now)
  \end{itemize}
  \medskip
  \begin{itemize}
    \item Planning to submit the paper to JEEM short 
    \item Max 3,000 words (but ``modestly longer papers will be considered'')
    \item My paper now has 3,200 words
  \end{itemize}
\end{frame}

\begin{frame}\frametitle{Data}
  \begin{itemize}
    \item Weather information from Japan Meteorological Agency
      \begin{itemize}
        \item Hourly information on temperature, rainfall, snowfall, snow on the ground
        \item I use average between 6 AM to 6 PM
      \end{itemize}
    \item Matriculation for UTokyo from University Basic Information
      \begin{itemize}
        \item Information from 2012 to 2020 on the number of matriculated students for each university by prefectures at which their high schools are located
        \item I calculate the matriculation shares for UTokyo from a prefecture $j$ in year $t$ as 
          \begin{equation*}
            \frac{\text{\# matriculations for UTokyo from $j$ in year $t$}}{\text{\# total matriculations for UTokyo in year $t$}}
          \end{equation*}
      \end{itemize}
  \end{itemize}
\end{frame}

\begin{frame}\frametitle{Summary Statistics}
  \begin{center}
    \begin{table}
      \caption{Summary Statistics}
      \footnotesize
      
\begin{tabular}[t]{lllllll}
\toprule
\multicolumn{1}{c}{ } & \multicolumn{1}{c}{N} & \multicolumn{1}{c}{Mean} & \multicolumn{1}{c}{SD} & \multicolumn{1}{c}{Median} & \multicolumn{1}{c}{Min} & \multicolumn{1}{c}{Max} \\
\cmidrule(l{3pt}r{3pt}){2-2} \cmidrule(l{3pt}r{3pt}){3-3} \cmidrule(l{3pt}r{3pt}){4-4} \cmidrule(l{3pt}r{3pt}){5-5} \cmidrule(l{3pt}r{3pt}){6-6} \cmidrule(l{3pt}r{3pt}){7-7}
Matriculation share (\%) & 517 & 2.08 & 5.22 & 0.76 & 0.06 & 37.35\\
Temperature (\degree C) & 423 & NA & NA & NA & NA & NA\\
Hourly precipitation (mm) & 423 & NA & NA & NA & NA & NA\\
Hourly snowfall (cm) & 423 & NA & NA & NA & NA & NA\\
Cumulated snow (cm) & 423 & NA & NA & NA & NA & NA\\
\bottomrule
\end{tabular}

    \end{table}
  \end{center}
\end{frame}

\begin{frame}\frametitle{Map of matriculation shares (\%)}
  \begin{center}
    \begin{figure}
      %\caption{Map of matriculation shares (\%)}
      \includegraphics[width=7.5cm]{../Output/images/admission_map.pdf}
      \tiny
      \begin{tablenotes}
      \item Notes:
        This map shows the matriculation shares (\%) for the University of Tokyo from each prefecture.
        The matriculation share is calculated based on the ratio of the number of students from high schools in the prefecture to the total number of matriculation.
        Average of the shares across years is shown in the map.
        Although not shown in the legend due to the lack of the space, the lightest color in the map means the matriculation share between 0 to 0.5\% 
      \end{tablenotes}
    \end{figure}
  \end{center}
\end{frame}

\begin{frame}\frametitle{Empirical Strategy}
  \begin{equation*}
    Y_{jt} = \sum_k \alpha_k T_{jt}^k + X_{jt}' \beta + \mu_j + \tau_t + \epsilon_{jt}.
  \end{equation*}
  \begin{itemize}
    \item $Y_{jt}$: matriculation share (\%) for UTokyo of a prefecture $j$ in year $t$
    \item $T_{jt}^k$: bin of average temperature across two exam dates in $j$ in year $t$
    \item $X_{jt}$: other weather variables such as rainfall, snowfall, and cumulated snow on the ground on exam dates
    \item $\mu_j$: prefecture fixed effects
    \item $\tau_t$: year fixed effects
    \item $\epsilon_{jt}$: error term
    \item Standard errors are clustered at the prefecture level
  \end{itemize}
\end{frame}

\begin{frame}\frametitle{Temperature variations in each prefecture}
  \begin{center}
    \begin{figure}
      %\caption{Map of matriculation shares (\%)}
      \includegraphics[width=9.5cm]{../Output/images/temperature_diff.pdf}
      \tiny
      \begin{tablenotes}
      \item Notes:
        Temperature at prefecture capitals between 6 AM and 6 PM on days of Center Test is used.
        Average across two test days in each year is shown in the figure.
        The prefectures are ordered by the mean temperature across years (lowest to left and highest to right).
      \end{tablenotes}
    \end{figure}
  \end{center}
\end{frame}

\begin{frame}\frametitle{Deviation of temperatures from within-prefecture averages}
  \begin{center}
    \begin{figure}
      %\caption{Map of matriculation shares (\%)}
      \includegraphics[width=7.5cm]{../Output/images/temperature_diff_by_year.pdf}
      \tiny
      \begin{tablenotes}
      \item Notes:
        Temperature at prefecture capitals between 6 AM and 6 PM on days of Center Test is used.
        Deviations of average temperatures across two test days from their within-prefecture means are shown in the figure by year.
        In each panel, red dots are the deviations in the year shown above and gray dots are the deviations in the other years.
        The prefectures are ordered by the mean temperature across years (lowest to left and highest to right).
      \end{tablenotes}
    \end{figure}
  \end{center}
\end{frame}

\begin{frame}\frametitle{Temperature effects on the matriculation shares (\%)}
  \begin{center}
    \begin{figure}
      \includegraphics[width=6.5cm]{../Output/images/main_reg_2.pdf}
      \tiny
      \begin{tablenotes}
      \item Notes:
        The figure shows the regression coefficients of the matriculation shares for UTokyo (\%) on temperature bins.
        I use temperature at prefecture capitals between 6 AM and 6 PM on days of Center Test.
        Each interval on the x-axis does not include the right end.
        The 90\% confidence intervals are shown.
        Prefecture fixed effects and year fixed effects are also included.
        Standard errors are clustered at the prefecture level.
      \end{tablenotes}
    \end{figure}
  \end{center}
\end{frame}

\begin{frame}\frametitle{Effects of other weather variables on the matriculation shares (\%)}
    \begin{table}[h]
      \Wider[4em]{
      \center
      \fontsize{6}{8}\selectfont
      
% Table created by stargazer v.5.2.3 by Marek Hlavac, Social Policy Institute. E-mail: marek.hlavac at gmail.com
% Date and time: Wed, Oct 25, 2023 - 20:16:18
\begin{tabular}{@{\extracolsep{5pt}}lccc} 
\\[-1.8ex]\hline 
\hline \\[-1.8ex] 
 & \multicolumn{3}{c}{\textit{Dependent variable:}} \\ 
\cline{2-4} 
\\[-1.8ex] & \multicolumn{3}{c}{Matriculation share (\%)} \\ 
\\[-1.8ex] & (1) & (2) & (3)\\ 
\hline \\[-1.8ex] 
 Rainfall (mm) & $-$0.02 & 0.06 & 0.06 \\ 
  & (0.09) & (0.08) & (0.08) \\ 
  & & & \\ 
 Snowfall (cm) & 0.08 &  &  \\ 
  & (0.10) &  &  \\ 
  & & & \\ 
 Cumulated snow (cm) &  & $-$0.002$^{**}$ &  \\ 
  &  & (0.001) &  \\ 
  & & & \\ 
 Cumulated snow $>$ 10 cm &  &  & $-$0.11$^{***}$ \\ 
  &  &  & (0.04) \\ 
  & & & \\ 
\hline \\[-1.8ex] 
Temperature bins & Yes & Yes & Yes \\ 
Prefecture FE & Yes & Yes & Yes \\ 
Year FE & Yes & Yes & Yes \\ 
Observations & 423 & 423 & 423 \\ 
\hline 
\hline \\[-1.8ex] 
\textit{Note:}  & \multicolumn{3}{r}{$^{*}$p$<$0.1; $^{**}$p$<$0.05; $^{***}$p$<$0.01} \\ 
\end{tabular} 

      \tiny
      \begin{tablenotes}
      \item 
        The outcome variable is the matriculation shares (\%) for the University of Tokyo from each prefecture.
        The matriculation share is calculated based on the ratio of the number of students from high schools in the prefecture to the total number of matriculation.
        I use weather variables at prefecture capitals between 6 AM and 6 PM on days of Center Test.
        Standard errors are clustered at the prefecture level.
      \end{tablenotes}
    }
    \end{table}
\end{frame}

\begin{frame}\frametitle{Placebo test (including temperature one year after the exams)}
  \begin{minipage}{0.49\textwidth}
    \begin{center}
      Contemporaneous temperature
    \end{center}
    \begin{figure}[h]
      \centering
      \includegraphics[width = \textwidth]{../Output/images/reg_placebo_exam_4.pdf}
    \end{figure}
  \end{minipage}
  \begin{minipage}{0.49\textwidth}
    \begin{center}
      Temperature one year later
    \end{center}
    \begin{figure}[h]
      \includegraphics[width = \textwidth]{../Output/images/reg_placebo_f1_4.pdf}
      \centering
    \end{figure}
  \end{minipage}
  \tiny
  \begin{tablenotes}
  \item Notes;
    The figures show the regression coefficients of the matriculation shares for UTokyo (\%) on temperature bins.
    In the left figure, the estimates on the contemporaneous temperature bins are shown.
    In the right figure, the estimates on the temperature one year later are shown.
    Rainfall and cumulated snow on the ground (both contemporaneous and one year later) are included in the regression.
    Each interval on the x-axis does not include the right end.
    The 90\% confidence intervals are shown.
    Prefecture fixed effects and year fixed effects are also included.
    Standard errors are clustered at the prefecture level.
  \end{tablenotes}
\end{frame}

\begin{frame}\frametitle{Cognitive performance at the exam vs. Preparation}
  \begin{minipage}{0.49\textwidth}
    \begin{center}
      Temperature at the exam dates
    \end{center}
    \begin{figure}[h]
      \centering
      \includegraphics[width = \textwidth]{../Output/images/reg_pre10_exam_4.pdf}
    \end{figure}
  \end{minipage}
  \begin{minipage}{0.49\textwidth}
    \begin{center}
      Average temperature for 10 days before the exam
    \end{center}
    \begin{figure}[h]
      \includegraphics[width = \textwidth]{../Output/images/reg_pre10_pre10_4.pdf}
      \centering
    \end{figure}
  \end{minipage}
  \tiny
  \begin{tablenotes}
  \item Notes;
    The figures show the regression coefficients of the matriculation shares for UTokyo (\%) on temperature bins.
    In the left figure, the estimates on the temperature on the exam dates are shown.
    In the right figure, the estimates on the average temperature for 10 days before the exam are shown.
    Rainfall and cumulated snow on the ground (both on the exam days and average for 10 days before the exam) are included in the regression.
    Each interval on the x-axis does not include the right end.
    The 90\% confidence intervals are shown.
    Prefecture fixed effects and year fixed effects are also included.
    Standard errors are clustered at the prefecture level.
  \end{tablenotes}
\end{frame}

\begin{frame}\frametitle{Effects of other weather variables on the matriculation shares (\%)}
  \begin{table}[h]
    \Wider[4em]{
    \center
    \fontsize{6}{7}\selectfont
    
% Table created by stargazer v.5.2.2 by Marek Hlavac, Harvard University. E-mail: hlavac at fas.harvard.edu
% Date and time: Thu, May 13, 2021 - 14:05:20
\begin{tabular}{@{\extracolsep{5pt}}lcccc} 
\\[-1.8ex]\hline 
\hline \\[-1.8ex] 
 & \multicolumn{4}{c}{\textit{Dependent variable:}} \\ 
\cline{2-5} 
\\[-1.8ex] & \multicolumn{4}{c}{Matriculation share (\%)} \\ 
\\[-1.8ex] & (1) & (2) & (3) & (4)\\ 
\hline \\[-1.8ex] 
 Rainfall (mm) (exam dates) &  &  & $-$0.02 & 0.06 \\ 
  &  &  & (0.10) & (0.08) \\ 
  & & & & \\ 
 Snowfall (cm) (exam dates) &  &  & 0.12 &  \\ 
  &  &  & (0.10) &  \\ 
  & & & & \\ 
 Cumulated snow (cm) (exam dates) &  &  &  & $-$0.002$^{*}$ \\ 
  &  &  &  & (0.001) \\ 
  & & & & \\ 
 Rainfall (mm) (previous 10 days) & $-$0.003 & $-$0.01 & $-$0.003 & $-$0.004 \\ 
  & (0.01) & (0.01) & (0.01) & (0.01) \\ 
  & & & & \\ 
 Snowfall (cm) (previous 10 days) & $-$0.02$^{*}$ &  & $-$0.02 &  \\ 
  & (0.01) &  & (0.01) &  \\ 
  & & & & \\ 
 Cumulated snow (cm) (previous 10 days) &  & $-$0.002$^{*}$ &  & 0.001 \\ 
  &  & (0.001) &  & (0.002) \\ 
  & & & & \\ 
\hline \\[-1.8ex] 
Temperature bins & Yes & Yes & Yes & Yes \\ 
Prefecture FE & Yes & Yes & Yes & Yes \\ 
Year FE & Yes & Yes & Yes & Yes \\ 
Observations & 423 & 423 & 423 & 423 \\ 
\hline 
\hline \\[-1.8ex] 
\textit{Note:}  & \multicolumn{4}{r}{$^{*}$p$<$0.1; $^{**}$p$<$0.05; $^{***}$p$<$0.01} \\ 
\end{tabular} 

    \tiny
    \begin{tablenotes}
    \item 
      The outcome variable is the matriculation shares (\%) for the University of Tokyo from each prefecture.
      The matriculation share is calculated based on the ratio of the number of students from high schools in the prefecture to the total number of matriculation.
      Standard errors are clustered at the prefecture level.
    \end{tablenotes}
  }
  \end{table}
\end{frame}

\begin{frame}\frametitle{Female vs. Male}
  \begin{minipage}{0.49\textwidth}
    \begin{center}
      Female \\
      (mean: 0.38 \%, median: 0.13 \%)
    \end{center}
    \begin{figure}[h]
      \centering
      \includegraphics[width = \textwidth]{../Output/images/reg_gender_2.pdf}
    \end{figure}
  \end{minipage}
  \begin{minipage}{0.49\textwidth}
    \begin{center}
      Male \\
      (mean: 1.70 \%, median: 0.61 \%)
    \end{center}
    \begin{figure}[h]
      \includegraphics[width = \textwidth]{../Output/images/reg_gender_4.pdf}
      \centering
    \end{figure}
  \end{minipage}
  \tiny
  \begin{tablenotes}
  \item Notes;
    The figures show the regression coefficients of the matriculation shares for UTokyo (\%) on temperature bins.
    In the left figure, the outcome is the matriculation share of female students from each prefecture.
    In the right figure, the outcome is the matriculation share of male students from each prefecture.
    Rainfall and cumulated snow on the ground (both on the exam days and average for 10 days before the exam) are included in the regression.
    Each interval on the x-axis does not include the right end.
    The 90\% confidence intervals are shown.
    Prefecture fixed effects and year fixed effects are also included.
    Standard errors are clustered at the prefecture level.
  \end{tablenotes}
\end{frame}

%\begin{frame}\frametitle{Humanity vs. Science}
%  \begin{minipage}{0.49\textwidth}
%    \begin{center}
%      Humanity \\
%      (mean: 2.07 \%, median: 0.76 \%)
%    \end{center}
%    \begin{figure}[h]
%      \centering
%      \includegraphics[width = \textwidth]{../Output/images/reg_major_2.pdf}
%    \end{figure}
%  \end{minipage}
%  \begin{minipage}{0.49\textwidth}
%    \begin{center}
%      Science \\
%      (mean: 2.09 \%, median: 0.76 \%)
%    \end{center}
%    \begin{figure}[h]
%      \includegraphics[width = \textwidth]{../Output/images/reg_major_4.pdf}
%      \centering
%    \end{figure}
%  \end{minipage}
%  \tiny
%  \begin{tablenotes}
%  \item Notes;
%    The figures show the regression coefficients of the matriculation shares for UTokyo (\%) on temperature bins.
%    In the left figure, the outcome is the matriculation share to humanity major from each prefecture.
%    In the right figure, the outcome is the matriculation share to science major from each prefecture.
%    Rainfall and cumulated snow on the ground (both on the exam days and average for 10 days before the exam) are included in the regression.
%    Each interval on the x-axis does not include the right end.
%    The 90\% confidence intervals are shown.
%    Prefecture fixed effects and year fixed effects are also included.
%    Standard errors are clustered at the prefecture level.
%  \end{tablenotes}
%\end{frame}



\end{document}

